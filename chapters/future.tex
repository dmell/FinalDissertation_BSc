\chapter{Testing and future developments}
\label{cha:future}

\textit{About what has been done, how it works. About what still needs to be developed: how to scale correctly the network, how to sell it. The proposed solution is more an "interface", a proof-of-concept. It can represent a good starting base for a custom project, responding to a client's needs. The company won't send the product as a whole, but more likely an ad hoc service. Some logistic companies already showed interest. Limits of the system as it is now: in its current shape, it is similar to a service like "Paypal", so a secure layer quite complicated and not perfect. Possible extensions: sending updated with the coordinates of the good, tracking more exchanges (e.g. not only the arrival of the good at a certain point, but the departure too)}

%hyperledger general purpose chaincode, MQTT vulnerabilities, dead messages