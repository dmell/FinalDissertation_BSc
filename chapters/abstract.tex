\chapter*{Abstract} % senza numerazione
\label{abstract}

\addcontentsline{toc}{chapter}{Abstract} % da aggiungere comunque all'indice

%\textit{Hints provided by the template:}
\iffalse



  Sommario è un breve riassunto del lavoro svolto dove si descrive l'obiettivo, l'oggetto della tesi, le 
metodologie e le tecniche usate, i dati elaborati e la spiegazione delle conclusioni alle quali siete arrivati.  

Il sommario dell’elaborato consiste al massimo di 3 pagine e deve contenere le seguenti informazioni:
\begin{itemize}
  \item contesto e motivazioni 
  \item breve riassunto del problema affrontato
  \item tecniche utilizzate e/o sviluppate
  \item risultati raggiunti, sottolineando il contributo personale del laureando/a
\end{itemize}

\fi

All logistic processes involve the exchange of goods between entities that do not fully trust each other. For example, let us consider a very common scenario: a delivery by a courier that includes an intermediate stop. This can be the porter of a building, or an authorized post office. What usually happens is that when the parcel arrives the receiver testifies with their signature (on paper or on an electronic device) that the exchange has occurred. However, this is not always true, because not all the deliveries include this possibility or because the persons directly performing the exchange do not fully respect the protocol. As a result, the process is vulnerable: a hand-written signature is often counterfeitable and when it is not even present, it is possible to generate a ``my word against yours'' debate.

This problem becomes worse when it includes more steps: this happens, for example, in complex systems of organizations, people and activities employed in the supply chain sector: here the handovers are even more articulated and it is important to guarantee a successful outcome of the processes, that are crucial for the companies involved. A series of signatures can be a solution, but, as aforementioned, several things can go wrong.

Who has not been taken into consideration yet is the final recipient. They know nothing about the events happening and they could only be notified autonomously by one of the entities involved, whom they would have to trust. 

Together with the \textsc{Spindox Labs} team, we tried to investigate a solution to this problem. We started from the wide experience that the company had with tracking devices, trying to understand if one of them could be applicable to this case. After a thorough analysis of the alternatives, we found a good solution, satisfying our needs. After this, the most significant part of our research involved the study on the architecture that our system required. We wanted to find an innovative solution that eliminated any centralized authority, because we could not introduce one in the problem: many entities at the same level, without a supervising one. 

The result of this work is a decentralized architecture approaching the problem with a combination of diverse technologies. We implemented a private blockchain, to guarantee distribution and reliability, we integrated IoT devices to track the deliveries and we included some web servers and interfaces to facilitate the usage of the service. This project has been a research activity, thus no specific requirements from any client have been considered. The solution is a ``proof-of-concept'', demonstrating the power of the technologies involved and it can represent a complete starting base for a customized version, tackling a particular problem.

The dissertation is organized as follows: Chapter \ref{cha:background} and \ref{cha:main} introduce the problem and present the architecture. Then the proposed solution is split into its three main components, namely tracking, decentralization and servers. Chapter \ref{cha:future} evaluates the project, outlining limits and possible extensions. Chapter \ref{cha:conclusion} concludes the thesis.

%\textit{Summary of contributions and results}

\newpage