\chapter{Background}
\label{cha:background}
\textit{Detailed explanation of the problem and its scope: starting from a practical description, a formalization of the entities involved and what they want to achieve. 2 examples (deliveries with couriers and port exchanges). Then, the intentions of the company I have worked for.}

All logistic processes involve the exchange of goods between entities that do not fully trust each other. For example, let us consider a very common scenario: a delivery by a courier that includes a intermediate stop. This can be the porter of a building, or an authorized post office. What usually happens is that when the parcel arrives the receiver testifies with their signature (on paper or on a digital device) that the exchange has occurred. However, this is not always true, because not all deliveries include this possibility or because the persons directly performing the exchange do not fully respect the protocol. As a result, this process is vulnerable: a hand-written signature is always counterfeitable and when it is not even present, it is possible to generate a ``my word against yours'' debate.

This problem becomes worse when it includes more steps: this happens, for example, in complex systems of organizations, people, activities employed in the supply chain sector: here the handovers are even more articulated and it is important to guarantee a successful outcome of the processes, that are crucial for the companies involved. A series of signatures can be a solution, but, as aforementioned, several things can go wrong.

Who has not been taken into consideration yet is the final recipient. They know nothing about the events happening, or at least, they can only be notified autonomously by one of the entities involved, whom they would have to trust. 

Formalizing, the typical scenario is the following. We have a sender \textit{S} and a final recipient \textit{R}. \textit{S} wants to send \textit{R} a particular good, \textit{G}. To perform the delivery, \textit{S} will designate a certain route that involves \textit{n} intermediate stops ($I_1, I_2, ..., I_n$) and will instruct \textit{m} dealers ($D_1, D_2, ..., D_m$) that will take care of each part of the delivery. From a theoretical point of view, nothing forbids to have two identical dealers $D_i$ and $D_j$. Indeed, as we will see later, we are not really interested in the dealers themselves, but more in the companies that these dealers work for. We assume that \textit{S} decides the route, or at least, knows the various stops needed. We would like to find a model able to register precisely each handover, by making use of secure techniques that do not allow counterfeiting or fraud: each entity should be protected against a potential malicious one. Finally, \textit{R} should be able to monitor the events involving \textit{G}, in order to check the progresses of the delivery.

During my internship at \textsc{Spindox Labs}, we tried to investigate a solution to this problem. 

\section{Decentralization}
\label{sec:decentralization}
Why decentralization can be the solution, how it differs from a traditional approach. The blockchain, with its hype, as a valid alternative. Deep explanation of what it is etc.

